\documentclass[12pt,a4paper, reqno]{amsart}
\usepackage[slovene]{babel}
\usepackage[utf8]{inputenc}
%\usepackage[T1]{fontenc}
\usepackage{amsmath,amssymb,amsfonts}
\usepackage[dvipsnames,usenames]{color}
\usepackage{algorithmicx,algpseudocode}
\usepackage{graphicx}
\usepackage{wrapfig}
\usepackage{varwidth}
\usepackage{caption}
\captionsetup{justification   = centering,
              singlelinecheck = false}

\textwidth 15cm
\textheight 24cm
\oddsidemargin.5cm
\evensidemargin.5cm
\topmargin-5mm
\addtolength{\footskip}{10pt}
\pagestyle{plain}

\overfullrule=15pt % oznaci predlogo vrstico

\newtheorem{definicija}{Definicija}[section]
\newtheorem{lema}[definicija]{Lema}
\newtheorem{opomba}[definicija]{Opomba}
\newtheorem{izrek}[definicija]{Izrek}
\newtheorem{trditev}[definicija]{Trditev}
\newtheorem{posledica}[definicija]{Posledica}

\let\oldref\ref
\renewcommand{\ref}[1]{(\oldref{#1})}

\def\R{\mathbb R}
\def\N{\mathbb N}
\def\Z{\mathbb Z}
\def\C{\mathbb C}
\def\Q{\mathbb Q}

\renewcommand{\algorithmicrequire}{{\bf Vhod:}}

\begin{document}

\thispagestyle{empty}
\noindent{\large
Univerza v Ljubljani \hfill Ljubljana, \today\\[1mm]
Fakulteta za matematiko in fiziko  \\[5mm]
%IŠRM -- 2.~stopnja 
}
\medskip
%\vfill
\begin{center}{\large
Računalniško podprto geometrijsko oblikovanje\\[4mm]
% Seminarska naloga\\[4mm]
{\bf Iskanje presečišč B\`{e}zierjevih krivulj z metodo hibridnih izrezkov}\\[4mm]
Domen Keglevič\\[6mm]
}
\end{center}
\medskip
% tu se zacne tekst seminarja
\section{Uvod}
V seminarski nalogi nas bo zanimalo kako najti presečišča dveh ravninskih B\`{e}zier\-jevih krivulj. Pri tem nas ne zanimajo koordinate presečišč, temveč točke v domeni, ki se preslikajo v presečišča. Ta problem lahko formalno opišemo na naslednji način. Naj bosta ${\bf f}:[\alpha,\beta]\rightarrow \R^2$ in ${\bf g}:[\xi,\eta]\rightarrow \R^2$ ravninski B\`{e}zierjevi krivulji. Želimo najti učinkovit algoritem, ki za poljuben $\epsilon >0$ vrne pare intervalov $[\alpha_i,\beta_i]$ in $[\xi_i,\eta_i]$, ki vsebujejo presečišča ${\bf f}(t_i) = {\bf g}(s_i)$, $t_i\in[\alpha_i,\beta_i]$, $s_i\in[\xi_i,\eta_i]$, tako da velja $|\alpha _i - \beta _i| < \epsilon$ in $|\xi _i - \eta _i| < \epsilon$.

Ta problem se da rešiti na več načinov. Tu bomo predstavili metodo hibridnih izrezkov, ki je razširitev metode B\`{e}zierjevih izrezkov. 

Metoda B\`{e}zierjevih izrezkov najprej aproksimira krivuljo ${\bf g}$ z ozkim pasom $\mathcal{L}$ v ravnini ({\em fat line}), tako da ${\bf g}$ leži znotraj $\mathcal{L}$. Nato izračuna presek pasa $\mathcal{L}$ in konveksne ovojnice kontrolnih točk krivulje ${\bf f}$, ter definicijsko območje krivulje ${\bf f}$ omeji na intervale, kjer je presek neprazen. 
Nato vlogi ${\bf f}$ in ${\bf g}$ zamenja in postopek ponavlja, dokler ne pride do željene natančnosti.

Ideja metode hibridnih izrezkov je podobna kot ideja metode B\`{e}zierjevih izrezkov, le da ne opazujemo presek pasa $\mathcal{L}$ s konveksno ovojnico kontrolnih točk krivulje ${\bf f}$, ampak z območjem $\mathcal{P}$ ({\em fat curve}). Tega dobimo tako, da z nižanjem stopnje krivulje ${\bf f}$ dobimo aproksimacijo, ki jo lahko premaknemo v dve nasprotni smeri tako, da je krivulja ${\bf f}$ vsebovana v območju $\mathcal{P}$ vmes (slika \ref{slika2}). Nato krivuljo ${\bf f}$ zmanjšamo na tiste intervale, kjer je presek $\mathcal{L}\cap\mathcal{P}$ neprazen. V naslednjem koraku vlogi ${\bf f}$ in ${\bf g}$ zamenjamo in postopek ponavljamo do željene natančnosti.
\begin{figure}[!h]
  \centering
    \includegraphics[width=0.40\linewidth]{2}
  	\caption{{\em Fat line $\mathcal{L}$} in {\em fat curve $\mathcal{P}$}, ki vsebujeta {\bf g} in {\bf f}.}
  	\label{slika2}
\end{figure}
\medskip

\section{Metoda hibridnih izrezkov}
Vpeljimo nekaj oznak, ki jih bomo uporabljali v nadaljevanju. Z 
\begin{equation*}
B_{i,[\alpha,\beta]}^n(t) = \binom{n}{i} \frac{(t-\alpha)^i(\beta - t)^{n-i}}{(\beta - \alpha)^n}
\end{equation*}
označimo $i$-ti Bernsteinov bazni polinom stopnje $n$ na intervalu $[\alpha,\beta]$. Intervale $[\alpha,\beta]$ namesto intervala $[0,1]$ opazujemo zato, ker bo algoritem generiral zaporedje vedno manjših intervalov.

\begin{trditev}
Velja $B_{i,[\alpha,\beta]}^n(t) \geq 0$ za vsak $t\in [\alpha,\beta]$.
\end{trditev}
\proof V definiciji $B_{i,[\alpha,\beta]}^n(t)$ so vsi členi v števcu in imenovalcu večji ali enaki $0$.
\endproof
\begin{trditev}
Velja $\sum _{i=0}^n{B_{i,[\alpha,\beta]}^n(t)} = 1$.
\end{trditev}
\proof
Vsoto razpišemo in dobimo
\begin{equation*}
\begin{split}
\sum _{i=0}^n{B_{i,[\alpha,\beta]}^n(t)} = &
\sum _{i=0}^n \binom{n}{i} \frac{(t-\alpha)^i(\beta - t)^{n-i}}{(\beta - \alpha)^n} \\
=& 
\frac{1}{(\beta - \alpha)^n}\sum _{i=0}^n \binom{n}{i}(t-\alpha)^i(\beta - t)^{n-i} \\
= &\frac{1}{(\beta - \alpha)^n} \cdot ((t - \alpha) + (\beta - t))^n = 1.
\end{split}
\end{equation*}
\endproof

Naj bosta ${\bf f}$ in ${\bf g}$ ravninski B\`{e}zierjevi krivulji dani z 
$${\bf f}(t) = \sum _{i=0}^{n} {\bf a}_i B_{i,[\alpha,\beta]}^n(t), \qquad t\in[\alpha,\beta]$$
in
$${\bf g}(s) = \sum _{j=0}^{m} {\bf b}_i B_{j,[\xi,\eta]}^m(s), \qquad s\in[\xi,\eta],$$
kjer so ${\bf a}_i, {\bf b}_j \in \R ^2$ kontrolne točke od ${\bf f}$ oz. ${\bf g}$. Naj bo $\lVert \,\cdot\,\rVert$ Evklidska norma na $\R^2$. Definiramo še naslednje norme:
\begin{enumerate}
\item Normalizirana $L_2$ norma
$$
\lVert {\bf f} \rVert _2^{[\alpha,\beta]} = \sqrt{\frac{1}{\beta -\alpha}\int _{\alpha}^{\beta}\lVert {\bf} f(t)\rVert ^2 dt},
$$
\item $L_{\infty}$ norma
$$
\lVert {\bf f} \rVert _\infty^{[\alpha,\beta]} = \max _{t\in[\alpha,\beta]} \lVert {\bf f}(t)\rVert,
$$
\item BB norma
$$
\lVert {\bf f} \rVert _{\text{BB}}^{[\alpha,\beta]} = \max _{i=0,\ldots,n} \lVert {\bf a}_i\rVert.
$$
\end{enumerate}
Prepričajmo se, da je $\lVert \,\cdot\, \rVert _{\text{BB}}^{[\alpha,\beta]}$ res norma. Velja $\lVert {\bf f} \rVert _{\text{BB}}^{[\alpha,\beta]} \geq 0$, saj je $\lVert {\bf a}_i\rVert \geq 0$. Velja 
\begin{equation*}
\lVert {\bf f} \rVert _{\text{BB}}^{[\alpha,\beta]} = 0  \Leftrightarrow 
\max _{i=0,\ldots,n} \lVert {\bf a}_i\rVert = 0 
 \Leftrightarrow \lVert {\bf a}_i\rVert = 0 \Leftrightarrow {\bf a}_i = {\bf 0 }
 \Leftrightarrow {\bf f} = {\bf 0}.
\end{equation*}
Homogenost sledi iz enakosti 
\begin{equation*}
\lVert \gamma\, {\bf f} \rVert _{\text{BB}}^{[\alpha,\beta]}  = 
\max _{i=0,\ldots,n} \lVert \gamma\, {\bf a}_i\rVert = 
 |\gamma\, | \max _{i=0,\ldots,n}  \lVert {\bf a}_i\rVert = 
 |\gamma\,| \lVert {\bf f} \rVert _{\text{BB}}^{[\alpha,\beta]},
\end{equation*}
trikotniška neenakost pa iz 
\begin{equation*}
\begin{split}
\lVert {\bf f} + {\bf g} \rVert _{\text{BB}}^{[\alpha,\beta]} &= 
\max _{i=0,\ldots,n} \lVert {\bf a}_i + {\bf b}_i \rVert 
\leq \max _{i=0,\ldots,n} \left ( \lVert {\bf a}_i\rVert + \lVert {\bf b}_i \rVert\right ) \\
& \leq \max _{i=0,\ldots,n} \lVert {\bf a}_i\rVert + 
\max _{i=0,\ldots,n} \lVert {\bf b}_i \rVert = 
\lVert {\bf f} \rVert _{\text{BB}}^{[\alpha,\beta]} +
\lVert {\bf g} \rVert _{\text{BB}}^{[\alpha,\beta]}.
\end{split}
\end{equation*}
Tudi za $\lVert \,\cdot\, \rVert _2^{[\alpha,\beta]}$ in $\lVert \,\cdot\, \rVert _\infty^{[\alpha,\beta]}$ se lahko hitro prepričamo, da ustrezata definiciji norme.

\subsection{{\em Fat line}}
Naj bo ${\bf n}$ normalni vektor, ki je pravokoten na ${\bf b}_m - {\bf b}_0$. Definiramo predznačeni razdalji 
\begin{gather*}
d_{\text{max}} = \max _{i=0,\ldots ,m} ({\bf n}\cdot ({\bf b}_i - {\bf b}_0)),\\
d_{\text{min}} = \min _{i=0,\ldots ,m} ({\bf n}\cdot ({\bf b}_i - {\bf b}_0)).
\end{gather*}
Množico $\mathcal{L}$, ki ga sestavljajo točke, ki so od premice ${\bf b}_0{\bf b}_m$ oddaljene kvečjemu za $d_{\text{min}}$ oz. $d_{\text{max}}$ imenujemo {\em fat line}. Krivulja ${\bf g}$ je vsebovana v $\mathcal{L}$, saj je v $\mathcal{L}$ vsebovana njena konveksna ovojnica.
\begin{figure}[!h]
    \centering 
    \includegraphics[width=0.5\textwidth]{fat_line}
    \caption{Krivulja {\bf g} in {\em fat line.}}
  	%\caption{{\em Fat line $\mathcal{L}$}.}
  	\label{slika3}
\end{figure}


\subsection{\em Fat curve}
Naj bo ${\bf \hat{p}}(t)$ polinom stopnje $k < n$, ki optimalno aproksimira ${\bf f}(t)$ glede na normo $L_2$. Konstruiramo ga s pomočjo nižanja stopnje krivulje ${\bf f}(t)$, kot je opisano v \textcolor{red}{referenca}. S pomočjo višanja stopnje mu lahko zvišamo stopnjo do $n$ in zapišemo
$$
{\bf \hat{p}}(t) = {\bf p}(t) = \sum _{i=0}^n {\bf c}_iB_{i,[\alpha,\beta]}^n(t),
$$
kjer so ${\bf c}_i$ nove kontrolne točke. Naj bo 
\begin{equation}\label{def_delta}
\delta = \lVert {\bf f}(t) - {\bf p}(t) \rVert _{BB}^{[\alpha,\beta]}.
\end{equation}
Velja naslednja ocena
$$
\lVert {\bf f}(t) - {\bf p}(t) \rVert = \lVert \sum _{i=0}^n({\bf a}_i-{\bf c}_i)B_{i,[\alpha,\beta]}^n(t) \rVert
\leq \sum _{i=0}^n\lVert{\bf a}_i-{\bf c}_i\rVert B_{i,[\alpha,\beta]}^n(t) \leq \delta.
$$
Sledi, da ${\bf f}(t)$ leži med ${\bf p}_1(t) = \hat{{\bf p}}(t) + \delta {\bf n}(t)$ in ${\bf p}_2(t) = \hat{{\bf p}}(t) - \delta {\bf n}(t)$, kjer je ${\bf n}$ normala pravokotna na ${\bf b}_m - {\bf b}_0$ (kot v {\em fat line}). Naj bo
\begin{gather*}
d(t) = {\bf n}\cdot({\bf f}(t) - {\bf b}_0), \, d_0(t)= {\bf n}\cdot ({\bf p}(t) - {\bf b}_0)\\
d_1(t) = {\bf n} ({\bf p}_1(t) - {\bf b}_0) = d_0(t) + \delta \\
d_2(t) = {\bf n} ({\bf p}_2(t) - {\bf b}_0) = d_0(t) - \delta.
\end{gather*}
Potem velja ocena
$$
|d(t)-d_0(t)|=|{\bf n}\cdot({\bf f}(t)-{\bf p}(t))|\leq \lVert {\bf n}\rVert\cdot \lVert {\bf f}(t) - {\bf p}(t) \rVert \leq \lVert {\bf f}(t) - {\bf p}(t) \rVert _{\infty}^{[\alpha,\beta]} \leq \delta.
$$
To pomeni, da $d(t)$ leži v pasu med $d_1(t)$ in $d_2(t)$ kot prikazuje slika \ref{slika4}.
\begin{figure}[!h]
    \centering 
    \includegraphics[width=0.6\textwidth]{dist}
    \caption{}
  	\label{slika4}
\end{figure}

\subsection{Iskanje intervalov}
Iz zgornjih dveh razdelkov vemo, da je predznačena razdalja točke ${\bf g}(t)$ od premice ${\bf b}_0{\bf b}_m$ vsebovana v intervalu $[d_{\text{min}}, d_{\text{max}}]$, in da je predznačena razdalja točke ${\bf f}(t)$ od premice ${\bf b}_0{\bf b}_m$ vsebovana v intervalu $[d_2(t), d_1(t)]$. Zato lahko tista območja, kjer je $d_1(t) < d_{\text{min}}$ in $d_2(t) > d_{\text{max}}$, zavržemo. Splošneje, če najdemo rešitve enačb
$$
d_1(t) = d_{\text{min}} \, \text{ in } \, d_2(t) = d_{\text{max}},
$$
potem lahko v domeni krivulje ${\bf f}$ najdemo iskane intervale $[\alpha _i, \beta _i]$, ki vsebujejo točke, ki se preslikajo v presečišča od ${\bf f}$ in ${\bf g}$.

V obeh zgoraj navedenih enačbah iščemo ničle polinoma. Če smo krivuljo ${\bf f}$ aproksimirali s krivuljo stopnje $2$ ali $3$, potem lahko ti dve enačbi rešimo analitično. To je v praksi najbolj učinkovita možnost.


\subsection{Psevdo koda algoritma}\text{}
\smallskip
\begin{small}
   \begin{algorithmic}[1]
	\Require $({\bf f},{\bf g}, [\alpha,\beta], [\xi,\eta], k)$ : ravninski B\`{e}zierjevi krivulji, njuni domeni in stopnja aproksimacijske krivulje

	\If{$|\alpha - \beta | < \epsilon$ in  $|\xi - \eta| < \epsilon$} \hfill ustavitveni pogoj
		\State \Return $[\alpha,\beta],[\xi,\eta]$
	\Else
		\If{$|\alpha - \beta | < |\xi - \eta|$} \hfill če ima {\bf f} manjšo domeno
			\State $HybridClip( {\bf g}, {\bf f}, [\xi,\eta],[\alpha,\beta],k)$ 
			\hfill zamenjamo vlogi ${\bf f}$ in ${\bf g}$
		\Else
			\State $L, C \gets $ {\em fat line}$({\bf g})$, {\em fat curve}$({\bf f})$ 
			\hfill aproksimiraj ${\bf f}$ in ${\bf g}$
			\State Najdi intervale $[\alpha _i,\beta _i]$, kjer je $L\cap C\neq \emptyset$
			\If{$l > 0$ in $\max _{i=1,\ldots ,l}\, \{ |\alpha _i - \beta _i|\} \geq 
				\frac{1}{2}|\alpha -\beta |$} \hfill  aproksimacija ni dobra
				\State \Return \begin{varwidth}[t]{\linewidth}  
					$HybridClip({\bf f},{\bf g},[\alpha,\frac{1}{2}(\alpha+\beta)],[\xi,\frac{1}{2}(\xi+\eta)],k)$\par $ 
        \hskip\algorithmicindent
					\cup \, HybridClip({\bf f},{\bf g},[\alpha,\frac{1}{2}(\alpha+\beta)],[\frac{1}{2}(\xi+\eta), \eta],k)$\par$
        \hskip\algorithmicindent
					\cup \, HybridClip({\bf f},{\bf g},[\frac{1}{2}(\alpha+\beta), \beta],[\frac{1}{2}(\xi+\eta), \eta],k)$\par$
        \hskip\algorithmicindent
					\cup \, HybridClip({\bf f},{\bf g},[\frac{1}{2}(\alpha+\beta), \beta],[\xi, \frac{1}{2}(\xi+\eta)],k)$
					\end{varwidth}
			\Else \hfill  aproksimacija je dobra
				\State $S\gets \emptyset$
				\For{$i=1,\ldots,l$} 
					\State $S\gets S\cup HybridClip({\bf f},{\bf g},[\alpha _i,\beta _i], [\xi,\eta],k)$
					\hfill rekurziven klic
				\EndFor
				\State \Return $S$ \hfill vrnemo rezultat
			\EndIf
		\EndIf
	\EndIf
   \end{algorithmic}
\end{small}
\smallskip

Klic funkcije $HybridClip({\bf f},{\bf g}, [\alpha,\beta],[\xi,\eta],k)$ vrne pare intervalov $[\alpha_i,\beta_i], [\xi_i,\eta_i]$, ki vsebujejo vsa presečišča in so manjši od predpisanega $\epsilon$. Lahko se zgodi, da med njimi vrne tudi par intervalov, kjer se krivulji ${\bf f }$ in ${\bf g}$ ne sekata, ampak le prideta blizu skupaj.



\section{Red konvergence}

\begin{trditev}\label{norm_invariance}

invarianca norm za afine transormacije 
\end{trditev}
\proof

\endproof

\begin{lema}\label{prvalema}
Naj bo ${\bf f}$ ravninska B\`{e}zierjeva krivulja in ${\bf p}$ njena optimalna $L_2$ aproksimacija stopnje $k$. Potem obstaja konstanta $C$, da za poljuben interval $[\alpha,\beta]\subseteq[0,1]$ velja $\lVert {\bf f} - {\bf p} \rVert _{BB}^{[\alpha,\beta]} \leq C |\alpha - \beta|^{k+1}$.
\end{lema}
\proof
Spomnimo se, da za poljubni normi $\lVert\, \cdot\, \rVert _1$ in $\lVert\, \cdot \,\rVert _2$ na končno dimenzionalnem vektorskem prostoru $V$ obstajata konstanti $0 < C_1 \leq C_2$, tako da je 
\begin{equation}\label{norm_equiv}
C_1 \lVert v\rVert _2 \leq \lVert v\rVert _1 \leq C_2 \lVert v\rVert _2, \; v\in V.
\end{equation}
Zato obstajata konstanti $D_1$ in $D_2$, da je $\lVert {\bf r}\rVert _{BB}^{[\alpha,\beta]} \leq D_1 \lVert {\bf r}\rVert _2^{[\alpha,\beta]}$ in $\lVert {\bf r}\rVert _{2}^{[\alpha,\beta]} \leq D_2 \lVert {\bf r}\rVert _\infty^{[\alpha,\beta]}$ za vsak ${\bf r}\in  \Pi _{[\alpha,\beta]}^n$. Pri tem konstanti $D_1$ in $D_2$ nista odvisni od intervala $[\alpha,\beta]$, saj so po trditvi \ref{norm_invariance} norme invariantne glede na afine transformacije.

Od tod sledi, da je 
$\lVert {\bf f} - {\bf p} \rVert _{BB}^{[\alpha,\beta]} \leq D_1 \lVert {\bf f} - {\bf p} \rVert _{2}^{[\alpha,\beta]}$. Naj bodo komponente ${\bf q}_\alpha$ Taylorjevi polinomi stopnje $k$ razviti okrog točke $t = \alpha$ za vsako komponento krivulje ${\bf f}$. Potem velja
$$
D_1 \lVert {\bf f} - {\bf p} \rVert _{2}^{[\alpha,\beta]} \leq D_1 \lVert {\bf f} - {\bf q}_\alpha \rVert _{2}^{[\alpha,\beta]},
$$
saj je ${\bf p}$ optimalna $L_2$ aproksimacija za ${\bf f}$. Iz \ref{norm_equiv} sledi, da je 
$
D_1 \lVert {\bf f} - {\bf q}_\alpha \rVert _{2}^{[\alpha,\beta]} \leq D_1D_2 \lVert {\bf f} - {\bf q}_\alpha \rVert _{\infty}^{[\alpha,\beta]}
$. Spomnimo se, da lahko razliko med ${\bf f}(t) - {\bf q}_\alpha(t)$ zapišemo v obliki
$$
{\bf f}(t) - {\bf q}_\alpha(t) = \frac{{\bf f}^{(k+1)}(t_o)}{(k+1)!}(t-\alpha)^{k+1},
$$
kjer je ${\bf f}^{(k+1)}$ $(k+1)$-vi odvod krivulje ${\bf f}$ in kjer vse člene opazujemo po komponentah. Od tod dobimo oceno
$$
D_1D_2 \lVert {\bf f} - {\bf q}_\alpha \rVert _{\infty}^{[\alpha,\beta]} \leq
\frac{\sqrt{2}}{(k+1)!}D_1D_2\max _{t\in [0,1]} \lVert {\bf f}^{(k+1)}(t_0) \rVert |\alpha - \beta|^{k+1}.
$$
\endproof

\begin{lema}\label{drugalema}
Naj bo ${\bf f}$ ravninska B\`{e}zierjeva krivulja stopnje $n$. Potem obstajajo konstante $C_{j}$, tako da za poljuben interval $[\alpha,\beta]\subseteq[0,1]$ in optimalno $L_2$ aproksimacijo ${\bf p}$ stopnje $k$ od ${\bf f}$ velja
$\lVert {\bf f}^{(j)} - {\bf p}^{(j)} \rVert _{\infty} \leq C_{j}|\alpha - \beta|^{k+1-j}$, za $j=0,1,\ldots,k$. 
\end{lema}
\proof
Definirajmo novo normo z naslednjim predpisom
$$
\lVert {\bf r}\rVert _{*}^{[\alpha,\beta]} = \lVert {\bf r}\rVert _\infty^{[\alpha,\beta]} + |\alpha - \beta|\lVert {\bf r}'\rVert _\infty^{[\alpha,\beta]} + \ldots + 
 |\alpha - \beta|^k\lVert {\bf r}^{(k)}\rVert _\infty^{[\alpha,\beta]}.
$$
To je res norma, saj je $\lVert \,\cdot\, \rVert _\infty$ norma. Po trditvi \ref{norm_invariance} in iz \textcolor{Red}{neka referenca} sledi, da obstaja konstanta $D_1$, da je 
$$
\lVert {\bf r}\rVert _{*}^{[\alpha,\beta]} \leq D_1 \lVert {\bf r}\rVert _{2}^{[\alpha,\beta]}.
$$
S pomočjo te ocene lahko zapišemo 
\begin{align*}
\lVert {\bf f} - {\bf p}\rVert _{*}^{[\alpha,\beta]}  =  \lVert {\bf f} - {\bf p}\rVert _\infty^{[\alpha,\beta]}  &+  |\alpha - \beta|\lVert {\bf f}' - {\bf p}'\rVert _\infty^{[\alpha,\beta]} + \ldots +\\ 
   & +  |\alpha - \beta|^k\lVert {\bf f}^{(k)} - {\bf p}^{(k)}\rVert _\infty^{[\alpha,\beta]} \leq 
 D_1 \lVert {\bf f} - {\bf p}\rVert _{2}^{[\alpha,\beta]}
\end{align*}
Sedaj podobno kot v lemi \ref{prvalema} s pomočjo Taylorjevega polinoma in izreka o ostanku ocenimo
\begin{align*}
D_1 \lVert {\bf f} - {\bf p}\rVert _{2}^{[\alpha,\beta]} \leq 
D_1 \lVert {\bf f} - {\bf q}_\alpha\rVert _{2}^{[\alpha,\beta]} \leq 
D_1D_2 \lVert {\bf f} - &{\bf q}_\alpha\rVert _{\infty}^{[\alpha,\beta]} \leq \\
\frac{\sqrt{2}}{(k+1)!}  D_1D_2\max _{t\in [0,1]}& \lVert {\bf f}^{(k+1)}(t_0) \rVert |\alpha - \beta|^{k+1}.
\end{align*}
\endproof

\begin{definicija}
%\in
%\in \Pi _{[\xi,\eta]}^m
Naj bosta ${\bf f}(t)$ in ${\bf g}(s)$ ravninski B\`{e}zierjevi krivulji s presečiščem ${\bf z}_0 = {\bf f}(t_0) = {\bf g}(s_0)$. Presečišče ${\bf z}_0$ imenujemo:
\begin{itemize}
	\setlength\itemsep{0.33em}
\item {\em transverzalno presečišče}, če je ${\bf f}'(t_0)\times {\bf g}'(s_0) \neq {\bf 0}$,
\smallskip
\item {\em tangentno presečišče}, če je ${\bf f}'(t_0)\times {\bf g}'(s_0) = {\bf 0}$ in ${\bf f}'(t_0)\neq 0$, ${\bf g}'(s_0)\neq 0$, in 
\smallskip
\item {\em degenerirano presečišče}, če je ${\bf f}'(t_0) = {\bf 0}$ ali ${\bf g}'(s_0) = {\bf 0}$.
\end{itemize}
\end{definicija}


\begin{trditev}
Naj imata Bezierjevi krivulji ${\bf f}$ in ${\bf g}$ transverzalno presečišče v ${\bf f}(t_0)={\bf g}(s_0)$. Potem obstajajo konstante $C_f, C_f', C_g$ in $C_g'$, da za dovolj velike $i\in \N$ velja
\begin{equation}\label{prva_neenakost}
|\alpha _{i+1} - \beta _{i+1}| \leq C_f |\alpha _{i} - \beta _{i}|^{k+1} + C_g|\xi _{i} - \eta _{i}|^2
\end{equation}
oz.
\begin{equation}\label{druga_neenakost}
|\xi _{i+1} - \eta _{i+1}| \leq C_f' |\alpha _{i} - \beta _{i}|^{2} + C_g'|\xi _{i} - \eta _{i}|^{k+1}.
\end{equation}
\end{trditev}

\proof
Dokazali bomo le neenakost \ref{prva_neenakost}, saj je dokaz za \ref{druga_neenakost} podoben. 

Naj bosta $[\alpha _i,\beta _i]$ in $[\xi _i,\eta _i]$ zaporedji intervalov, ki jih algoritem generira. Oglejmo si kako se algoritem obnaša za velike $i$. Ker se dolžine intervalov v vsakem koraku zmanjšajo vsaj za polovico, je 
\begin{equation*}
\lim _{i\rightarrow \infty} |\alpha _i-\beta _i| = 0 \,\text{ in }\, \lim _{i\rightarrow \infty} |\xi _i - \eta _i| = 0.
\end{equation*}
 Sledi, da ${\bf b}_0{\bf b}_m$ konvergira proti tangenti ${\bf g}'(s_0)$. Zato gre normala ${\bf n}$ na ${\bf b}_0{\bf b}_m$ proti normali ${\bf n}_0$ na ${\bf g}(s_0)$.

Naj bo $\omega = {\bf n}_0\cdot {\bf f}'(t_0)$. Po predpostavki je ${\bf f}'(t_0)\times {\bf g}'(s_0) \neq {\bf 0}$, torej je $\omega \neq 0$.

Oglejmo si situacijo v $i$-tem koraku algoritma, ki jo prikazuje slika \textcolor{red}{\ref{slika1}}. Velja 
$$
|\alpha _{i+1}-\beta _{i+1}| = h_{i+1,{\bf f}} \leq L_{i+1} = l_{i+1,1} + l_{i+1,2} + l_{i+1,3}.
$$
Želimo oceniti člene na desni strani neenakosti. Ker je $\frac{\omega}{4} = \frac{d_{max}-d_{min}}{l_{i+1,1}  + l_{i+1,3}}$ je
\begin{equation}
l_{i+1,1}  + l_{i+1,3} = \frac{4(d_{max} - d_{min})}{\omega}.
\end{equation}
Želimo oceniti še člen $l_{i+1,2}$. V ta namen si najprej oglejmo ali sta funkciji $d_1(t)$ in $d_2(t)$ naraščajoči. Najprej opazimo, da obstaja tak $\epsilon _1>0$, da je za dovolj velike $i$  
\begin{equation}\label{prva_ocena}
|d'(t_0) - \omega| = |{\bf n}\cdot {\bf f}'(t_0) - {\bf n}_0 \cdot {\bf f}'(t_0)| < \frac{\omega}{4},
\end{equation}
ko je $|\xi _i - \eta _i| < \epsilon _1$, saj gre ${\bf n}$ proti ${\bf n}_0$, ko $i\rightarrow\infty$. Obstaja tudi tak $\epsilon _2 > 0$, da je za dovolj velike $i$
\begin{equation}\label{druga_ocena}
\lVert {\bf f}'(t) - {\bf f}'(t_0)\rVert < \frac{\omega}{4},
\end{equation}
ko je $|\alpha _i - \beta _i| < \epsilon _2$ saj je ${\bf f}'$ zvezna. Dalje obstaja tudi tak $\epsilon _3 > 0$, da za dovolj velike $i$ velja
\begin{equation}\label{tretja_ocena}
|d'(t)-d'_1(t)| < \frac{\omega}{4},
\end{equation}
ko je $|\alpha _i - \beta _i| < \epsilon _3$, saj po lemi \ref{drugalema} velja
$$
|d'(t)-d'_1(t)| = |{\bf n}\cdot ({\bf f}'(t) - {\bf p}'(t))| \leq \lVert {\bf f}'(t) - {\bf p}'(t) \rVert \leq \lVert {\bf f}'(t) - {\bf p}'(t) \rVert _{\infty}^{[\alpha,\beta]} \leq C|\alpha _i- \beta_i|^k.
$$
Naj bo $\epsilon _4 = \min (\epsilon _1, \epsilon _2, \epsilon _3)$. S pomočjo ocen \ref{prva_ocena} in \ref{druga_ocena} lahko ocenimo
\begin{equation*}
\begin{split}
|d'(t)-\omega| & = |{\bf n}\cdot {\bf f}'(t) - {\bf n}_0\cdot{\bf f}'(t_0)| \\
& \leq |{\bf n}\cdot {\bf f}'(t) - {\bf n}\cdot{\bf f}'(t_0)| + 
|{\bf n}\cdot {\bf f}'(t_0) - {\bf n}_0\cdot{\bf f}'(t_0)|\\
&\leq \lVert {\bf f}'(t) - {\bf f}'(t_0)\rVert + |d'(t_0)-\omega|\\
&\leq \frac{\omega}{4} + \frac{\omega}{4} = \frac{\omega}{2},
\end{split}
\end{equation*}
za vse dovolj velike $i$, tako da je $|\alpha _i-\beta _i| < \epsilon _4$ in $|\xi _i - \eta _i| < \epsilon _4$. Zato je 
\begin{equation}\label{ocena_omega}
d'(t)<\frac{\omega}{2},
\end{equation}
 in skupaj z \ref{tretja_ocena} dobimo, da je $d_1'(t) = d_2'(t) > \frac{\omega}{4}$. Torej smo ugotovili, da sta funkciji $d_1(t)$ in $d_2(t)$ strogo naraščajoči na dovolj poznih intervalih $[\alpha _i, \beta _i]$.

Od tod sledi, da za poljuben $y_0$, za katerega velja $d_1(\alpha _i) < y_0 < d_2(\beta _i)$, velja, da imata enačbi $d_1(t)=y_0$ in $d_2(t) = y_0$ rešitvi $t_1$ in $t_2$ na intervalu $[\alpha _i, \beta _i]$. Zato je
$$
l_{i+1, 2} \leq \sup _{y_0\in (d_1(\alpha _i), d_2(\beta _i))} \{ |t_1-t_2|\, ; \, d_1(t_1)=d_2(t_2)=y_0\}.
$$
Če najdemo oceno za $|t_1-t_2|$, potem lahko s pomočjo zgornje enačbe ocenimo $l_{i+1,2}$. Oceno za $|t_1-t_2|$ bomo dobili na sledeč način. 
Opazimo, da za $d_1(t_1) = d_2(t_2) = y_0$ velja
$$
{\bf n}\cdot ({\bf p} (t_1) - {\bf p}(t_2)) = 2\delta,
$$
kjer je $\delta$ kot v \textcolor{red}{delta}. Naj bo ${\bf p}(t)=(x(t),y(t))$. Potem po Lagrange-ovem izreku o srednji vrednosti obstajata takšna $t^{\star}$ in $t^{\diamond}$, da je
$$
{\bf p}(t_1) - {\bf p}(t_2) = (x(t_1) - x(t_2), y(t_1) - y(t_2)) = (x'(t^{\star})(t_1-t_2),y'(t^{\diamond})(t_1-t_2)).
$$
Od tod dobimo oceno
\begin{equation*}
\begin{split}
|{\bf n}\cdot(x'(t^{\star}), &y'(t^\diamond)) - d'(t)|  \\
& = |{\bf n} (x'(t^\star), y'(t^\diamond)) - {\bf n}\cdot {\bf f}'(t) | \\
& \leq \lVert (x'(t^\star), y'(t^\diamond)) - {\bf f}'(t)\rVert \\
& = \lVert (x'(t^\star),y'(t^\star)) - {\bf f}'(t) + (0, y'(t^\diamond)) - (0,y'(t^\star))\rVert \\
& \leq \lVert {\bf p}'(t^\star) - {\bf f}'(t)\rVert + \lVert {\bf p}'(t^\diamond) - {\bf p}'(t^\star)\rVert \\ 
& \leq \lVert {\bf p}'(t^\star) - {\bf p}'(t)\rVert + \lVert {\bf p}'(t) - {\bf f}'(t)\rVert + 
\lVert {\bf p}'(t^\diamond) - {\bf p}'(t^\star)\rVert \\
& \leq 2\max _{t^1,t^2\in [\alpha _i,\beta _i]} \lVert {\bf p}'(t^1) - {\bf p}'(t^2)\rVert + 
\lVert {\bf p}'(t) - {\bf f}'(t)\rVert _{\infty}^{[\alpha _i,\beta _i]}.
\end{split}
\end{equation*}
Po lemi \ref{drugalema} in ker je ${\bf p'}(t)$ enakomerno zvezna na $[\alpha _i,\beta _i]$ obstaja $\epsilon _5 > 0$, da za dovolj velike $i$ velja 
\begin{gather*}
\max _{t^1,t^2\in [\alpha _i,\beta _i]} \lVert {\bf p}'(t^1) - {\bf p}'(t^2)\rVert < \frac{\omega}{16},\\
\lVert {\bf p}'(t) - {\bf f}'(t)\rVert _{\infty}^{[\alpha _i,\beta _i]} < \frac{\omega}{8},
\end{gather*}
ko je $|\alpha _i - \beta _i| < \epsilon _5$. Naj bo $\epsilon _0 = \min (\epsilon _4, \epsilon _5)$. Potem je 
$$
|{\bf n}\cdot (x'(t^\star), y'(t^\diamond)) - d'(t)| < \frac{\omega}{4},
$$
ko je $|\alpha _i - \beta _i| < \epsilon _0$ in $|\xi _i - \eta _i| < \epsilon _0$.
Če \ref{ocena_omega} kombiniramo z zgornjo neenakostjo dobimo oceno
$$
{\bf n}\cdot (x'(t^\star), y'(t^\diamond)) > \frac{\omega}{4}.
$$
Ker je $2\delta _i = {\bf n}\cdot({\bf p}(t_1) - {\bf p}(t_2)) = (t_1-t_2){\bf n}\cdot(x'(t^\star),y'(t^\diamond))$, dobimo
$$
|t_1-t_2| < \frac{2\delta _i}{\omega /4 } = \frac{8\delta _i}{\omega}.
$$


\endproof
\begin{wrapfigure}{r}{\linewidth}
%\begin{figure}
  \begin{center}
    \includegraphics[width=0.7\linewidth]{1}
  \caption{Situacija v $i$-tem koraku algoritma.}
  \end{center}
%  \end{figure}
\label{slika1}
\end{wrapfigure}
\section{Eksperimentalni rezultati}
V \cite{hyb_clip} so opisani eksperimentalni testi metode hibridnih izrezkov. Implementirani sta aproksimaciji s krivuljo stopnje $2$ in $3$ ter narejene primerjave z metodo B\`{e}zierjevih izrezkov za vse tri različne tipe presečišč in za krivulji z več kot enim presečiščem. Izkaže se, da je tudi v praksi konvergenca metode hibridnih izrezkov boljša. Pri tem je aproksimacija s krivuljo stopnje $2$ pogosto boljša kot aproksimacija s krivuljo stopnje $3$, razlog pa je verjetno računsko manj zahtevna aproksimacija. Primerjave so prikazane na sliki~\ref{fig:primeri} (povzeto po \cite{hyb_clip}).

\begin{figure}[!h]
  \begin{center}
    \includegraphics[width=0.9\linewidth]{primeri}
  \caption{Primerjave časovne zahtevnosti.}
  \label{fig:primeri}
  \end{center}
  
\end{figure}

\section{Zaključek}
Predstavili smo metodo hibridnih izrezkov, ki omogoča izračun presečišč dveh ravninskih B\`{e}zierjevih krivulj. Algoritem temelji na aproksimaciji ene in druge krivulje z območjema, ki krivulji vsebujeta in zmanjšanjem domene le na tiste dele, kjer je presek območij neprazen. Na ta način dobimo zaporedje vedno manjših intervalov, ki konvergirajo proti presečiščem.

Pokazali smo, da je v primeru transverzalnih presečišč konvergenca vsaj kvadratična. Tega nismo mogli pokazati za primer tangentnih in degeneriranih presečišč. V primeru tangentnih presečišč algoritem degenerira v metodo {\em deli in vladaj}, saj območja ne omejijo krivulj dobro. V primeru degeneriranih presečišč je aproksimacija boljša.
Podobno sliko pokažejo tudi eksperimentalni rezultati. Hitrost konvergence transverzalnih presečišč je najboljša, hitro sledijo degenerirana presečišča in precej slabše se odrežejo tangentna presečišča.

Zaključimo, da je metoda hibridnih izrezkov uporabna tudi v praksi, saj najde vsa presečišča in je hitrejša od metode B\`{e}zierjevih izrezkov. Možnosti za razširitve metode hibridnih izrezkov vključujejo izračun presečišč dveh racionalnih B\`{e}zierjevih krivulj in izračun preseka dveh B\`{e}zierjevih ploskev.

% seznam uporabljene literature
\begin{thebibliography}{99}

\bibitem{bez_clip}
 Sederberg T., Nishita T., {\em Curve intersection using B\`{e}zier clipping}, Comput.
Aided Des., 1990, 538--49.

\bibitem{hyb_clip}Qi Lou, Ligang Liu, {\em Curve intersection using hybrid clipping}, Computers and
Graphics, 36 (5), 2012, 309-320.

\bibitem{ls_sq}Zhang R., Wang G., {\em Constrained B\`{e}zier curves best multi-degree reduction in
the $L_2$-norm}, Prog. Nat. Sci., 2005, 15(9): 843–50

%\bibitem{oznaka-enote-za-sklic}
%\textcolor{Red}{I.~Priimek, {\em Naslov "clanka}, okraj"sano ime revije {\bf letnik revije} (leto %izida) strani od--do.}
%
%\bibitem{navodilaOMF}
%\textcolor{Red}{C.~Velkovrh, {\em Nekaj navodil avtorjem za pripravo rokopisa}, Obzornik mat.\ fiz.\ {\bf 21} (1974) 62--64.}

\end{thebibliography}

\end{document}

